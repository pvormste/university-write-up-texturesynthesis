\section{Summary and Conclusion}
Now we have learned that optimized solutions for obtaining smaller representations are the bidirectional similarity and inverse texture synthesis. Compared to image resizing and croping they do not lose valuable information or image quality.\\
Bidirectional similarity works by gradually resizing the target image and recalculating in every step the pixels by comparing for completeness and coherence. It is also capable of importance weighting to mark areas in an image which should not be touched by the algorithm. The performance of the algorithm is slow but it can be also used to summarize videos or montage images and videos.\\
Inverse texture synthesis constantly changes the output image when recalculating the image by forward and inverse M-steps. Like bidirectional similarity this is a two-way-comparison. Also used are control maps and orientation fields to manipulate the final image result. The performance of this algorithm is fairly good and it can also handle forward synthesis. On the other side, the algorithm is limited when the control map is not reasonably enoutgh  related to the original image.

\subsection*{Conclusion}
When looking on both algorithms it can be observed that both are working nearly the same way. Although the operations do have different names (forward M-step instead of coherence and inverse M-step instead of completeness), they work similar by comparing patches in both ways.\\
Both algorithms achieve good results and both can be used in different manners like bidirectional similarity for videos and inverse texture synthesis for GPU-accelerated forward synthesis.\\
Inverse texture synthesis may be faster than bidirectional similarity but faster is not better if the algorithm is not applicable on the problem.\\
So it is not possible to say that one algorithm is better than the other one. For different cases one of the algorithms may be the better choice than the other one.